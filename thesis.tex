% Copyright (c) 2008-2009 solvethis
% Copyright (c) 2010-2016 Casper Ti. Vector
% Public domain.
%
% 使用前请先仔细阅读 pkuthss 和 biblatex-caspervector 的文档,
% 特别是其中的 FAQ 部分和用红色强调的部分。
% 两者可在终端/命令提示符中用
%   texdoc pkuthss
%   texdoc biblatex-caspervector
% 调出。

% 采用了自定义的(包括大小写不同于原文件的)字体文件名,
% 并改动 ctex.cfg 等配置文件的用户请自行加入 nofonts 选项;
% 其它用户不用加入 nofonts 选项,加入之后反而会产生错误。
\documentclass[UTF8]{pkuthss}

% 使用 biblatex 排版参考文献,并规定其格式(详见 biblatex-caspervector 的文档)。
% 这里按照英文文献在前,中文文献在后排序(“sorting = ecnty”);
% 若需按照中文文献在前,英文文献在后排序,请设置“sorting = centy”;
% 若需按照引用顺序排序,请设置“sorting = none”。
% 若需在排序中实现更复杂的需求,请参考 biblatex-caspervector 的文档。
\usepackage[backend = biber, style = caspervector, utf8, sorting = none]{biblatex}
\usepackage{indentfirst}
\usepackage{url}
\setlength{\parindent}{2em}
\usepackage{graphics}
\usepackage{graphicx}
\usepackage{booktabs}
% 按学校要求设定参考文献列表中的条目之内及之间的距离。
\setlength{\bibitemsep}{3bp}
% 对于 linespread 值的计算过程有兴趣的同学可以参考 pkuthss.cls。
\renewcommand*{\bibfont}{\zihao{5}\linespread{1.27}\selectfont}

% 设定文档的基本信息。
\pkuthssinfo{
	cthesisname = {硕士研究生学位论文}, ethesisname = {Doctor Thesis},
	ctitle = {新型微结构气体探测器性能模拟研究及GE1/1探测器生产QC工艺研究}, etitle = {Simulation study of the performance of new micropattern gaseous detectors 
	and the  Quality Control process for the production of GE1/1 detectors},
	cauthor = {何少坤},
	eauthor = {Shaokun He},
	studentid = {1501210102},
	date = {2018年6月},
	%date = {二$\bigcirc$一八年六月},
	school = {物理学院},
	cmajor = {粒子物理与原子核物理}, emajor = {Particle Physics and Nuclear Physics},
	direction = {中高能与粒子物理},
	cmentor = {班勇教授}, ementor = {Prof.\ Yong Ban},
	ckeywords = {蒙特卡洛模拟,微结构气体探测器,时间分辨率,位置分辨率,质量控制}, ekeywords = {Monte Carlo simulation, Micropattern gaseous detectors, Time resolution, Spatial resolution, Quality control}
}
% 载入参考文献数据库(注意不要省略“.bib”)。
\addbibresource{thesis.bib}

% 普通用户可删除此段,并相应地删除 chap/*.tex 中的
% “\pkuthssffaq % 中文测试文字。”一行。
\usepackage{color}
\def\pkuthssffaq{%
	\emph{\textcolor{red}{pkuthss 文档模版最常见问题:}}

	\texttt{\string\cite}、\texttt{\string\parencite} %
	和 \texttt{\string\supercite} 三个命令分别产生%
	未格式化的、带方括号的和上标且带方括号的引用标记:%
	\cite{test-en},\parencite{test-zh}、\supercite{test-en, test-zh}。

	若要避免章末空白页,请在调用 pkuthss 文档类时加入 \texttt{openany} 选项。

	如果编译时不出参考文献,
	请参考 \texttt{texdoc pkuthss}“问题及其解决”一章
	“上游宏包可能引起的问题”一节中关于 biber 的说明。
}

\begin{document}
	% 以下为正文之前的部分,默认不进行章节编号。
	\frontmatter
	% 此后到下一 \pagestyle 命令之前不排版页眉或页脚。
	\pagestyle{empty}
	% 自动生成封面。
	\maketitle
	% 版权声明。封面要求单面打印,故需新开右页。
	\cleardoublepage
	\include{chap/copyright}

	% 此后到下一 \pagestyle 命令之前正常排版页眉和页脚。
	\cleardoublepage
	\pagestyle{plain}
	% 重置页码计数器,用大写罗马数字排版此部分页码。
	\setcounter{page}{0}
	\pagenumbering{Roman}
	% 中英文摘要。
	\include{chap/abstract}
	% 自动生成目录。
	\tableofcontents

	% 以下为正文部分,默认要进行章节编号。
	\mainmatter
	%% 序言。
	%\include{chap/introduction}
	% 各章节。
	%\include{chap/abstract}
	\include{chap/chap1}
	\include{chap/chap2}
	\include{chap/chap3}
	%\include{chap/chap4}
	\include{chap/chap5}
	% 结论。
	\include{chap/conclusion}

	% 正文中的附录部分。
	%\appendix
	% 排版参考文献列表。bibintoc 选项使“参考文献”出现在目录中;
	% 如果同时要使参考文献列表参与章节编号,可将“bibintoc”改为“bibnumbered”。
	\printbibliography[title = {文献列表},heading = bibintoc]
	% 各附录。
	%\include{chap/encl1}

	% 以下为正文之后的部分,默认不进行章节编号。
	\backmatter
	\include{chap/achivement}
	% 致谢。
	\include{chap/acknowledge}
	% 原创性声明和使用授权说明。
	\include{chap/originauth}
\end{document}

% vim:ts=4:sw=4
